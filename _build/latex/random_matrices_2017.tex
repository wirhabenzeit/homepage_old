%% Generated by Sphinx.
\def\sphinxdocclass{article}
\IfFileExists{luatex85.sty}
 {\RequirePackage{luatex85}}
 {\ifdefined\luatexversion\ifnum\luatexversion>84\relax
  \PackageError{sphinx}
  {** With this LuaTeX (\the\luatexversion),Sphinx requires luatex85.sty **}
  {** Add the LaTeX package luatex85 to your TeX installation, and try again **}
  \endinput\fi\fi}
\documentclass[letterpaper,10pt,english]{sphinxhowto}
\ifdefined\pdfpxdimen
   \let\sphinxpxdimen\pdfpxdimen\else\newdimen\sphinxpxdimen
\fi \sphinxpxdimen=.75bp\relax
\ifdefined\pdfimageresolution
    \pdfimageresolution= \numexpr \dimexpr1in\relax/\sphinxpxdimen\relax
\fi
%% let collapsible pdf bookmarks panel have high depth per default
\PassOptionsToPackage{bookmarksdepth=5}{hyperref}
%% turn off hyperref patch of \index as sphinx.xdy xindy module takes care of
%% suitable \hyperpage mark-up, working around hyperref-xindy incompatibility
\PassOptionsToPackage{hyperindex=false}{hyperref}
%% memoir class requires extra handling
\makeatletter\@ifclassloaded{memoir}
{\ifdefined\memhyperindexfalse\memhyperindexfalse\fi}{}\makeatother

\PassOptionsToPackage{booktabs}{sphinx}
\PassOptionsToPackage{colorrows}{sphinx}

\PassOptionsToPackage{warn}{textcomp}

\catcode`^^^^00a0\active\protected\def^^^^00a0{\leavevmode\nobreak\ }
\usepackage{cmap}
\usepackage{fontspec}
\defaultfontfeatures[\rmfamily,\sffamily,\ttfamily]{}
\usepackage{amsmath,amssymb,amstext}
\usepackage{polyglossia}
\setmainlanguage{english}



\setmainfont{FreeSerif}[
  Extension      = .otf,
  UprightFont    = *,
  ItalicFont     = *Italic,
  BoldFont       = *Bold,
  BoldItalicFont = *BoldItalic
]
\setsansfont{FreeSans}[
  Extension      = .otf,
  UprightFont    = *,
  ItalicFont     = *Oblique,
  BoldFont       = *Bold,
  BoldItalicFont = *BoldOblique,
]
\setmonofont{FreeMono}[
  Extension      = .otf,
  UprightFont    = *,
  ItalicFont     = *Oblique,
  BoldFont       = *Bold,
  BoldItalicFont = *BoldOblique,
]



\usepackage[Bjarne]{fncychap}
\usepackage[,numfigreset=2,mathnumfig]{sphinx}

\fvset{fontsize=\small}
\usepackage{geometry}

\usepackage{sphinxcontribtikz}

% Include hyperref last.
\usepackage{hyperref}
% Fix anchor placement for figures with captions.
\usepackage{hypcap}% it must be loaded after hyperref.
% Set up styles of URL: it should be placed after hyperref.
\urlstyle{same}


\usepackage{sphinxmessages}




\title{Random Matrices}
\date{Oct 12, 2023}
\release{}
\author{Dominik Schröder}
\newcommand{\sphinxlogo}{\vbox{}}
\renewcommand{\releasename}{}

\begin{document}

\pagestyle{empty}
\sphinxmaketitle
\pagestyle{plain}
\sphinxtableofcontents
\pagestyle{normal}
\phantomsection\label{\detokenize{teaching/random_matrices_2017::doc}}


\sphinxAtStartPar
Introductory course on random matrices from \sphinxstyleemphasis{October 10 to November 23, 2017} at \sphinxstyleemphasis{IST Austria}. The course instructor is \sphinxhref{http://ist.ac.at/en/research/research-groups/erdoes-group/}{László Erdős} and the teaching assistant is \sphinxhref{http://ist.ac.at/~dschroed/}{Dominik Schröder}.


\section{Description}
\label{\detokenize{teaching/random_matrices_2017:description}}
\sphinxAtStartPar
Random matrices were first introduced in statistics in the 1920’s, but they were made famous by Eugene Wigner’s revolutionary vision. He predicted that spectral lines of heavy nuclei can be modelled by the eigenvalues of random symmetric matrices with independent entries (Wigner matrices). In particular, he conjectured that the statistics of energy gaps is given by a universal distribution that is independent of the detailed physical parameters. While the proof of this conjecture for realistic physical models is still beyond reach, it has recently been shown that the gap statistics of Wigner matrices is independent of the distribution of the matrix elements. Students will be introduced to the fascinating world of random matrices and presented with some of the basic tools for their mathematical analysis in this course.


\section{Target audience}
\label{\detokenize{teaching/random_matrices_2017:target-audience}}
\sphinxAtStartPar
Students with orientation in mathematics, theoretical physics, statistics and computer science. No physics background is necessary. Calculus, linear algebra and some basic familiarity with probability theory is expected.


\section{Evaluation}
\label{\detokenize{teaching/random_matrices_2017:evaluation}}
\sphinxAtStartPar
The final grade will be obtained as a combination of the student’s performance on the example sheets and an oral exam.


\section{Credits}
\label{\detokenize{teaching/random_matrices_2017:credits}}
\sphinxAtStartPar
3 ECTS


\section{Lecture notes}
\label{\detokenize{teaching/random_matrices_2017:lecture-notes}}
\sphinxAtStartPar
Related \DUrole{xref,download,myst}{notes} from the recent \sphinxhref{https://pcmi.ias.edu/program-index/2017ss}{PCMI summer school} on random matrices.


\section{Schedule}
\label{\detokenize{teaching/random_matrices_2017:schedule}}
\sphinxAtStartPar
The course lasts from October 10 – November 23, 2017.


\begin{savenotes}\sphinxattablestart
\sphinxthistablewithglobalstyle
\centering
\begin{tabulary}{\linewidth}[t]{TTTTT}
\sphinxtoprule
\sphinxstyletheadfamily 
\sphinxAtStartPar
\sphinxstylestrong{Day}
&
\sphinxAtStartPar

&\sphinxstyletheadfamily 
\sphinxAtStartPar
\sphinxstylestrong{Time}
&
\sphinxAtStartPar

&\sphinxstyletheadfamily 
\sphinxAtStartPar
\sphinxstylestrong{Room}
\\
\sphinxmidrule
\sphinxtableatstartofbodyhook
\sphinxAtStartPar
\sphinxstyleemphasis{Oct 12}
&
\sphinxAtStartPar
\sphinxstyleemphasis{Thu}
&
\sphinxAtStartPar
\sphinxcode{\sphinxupquote{11.20am–12.35pm}}
&
\sphinxAtStartPar
Lecture
&
\sphinxAtStartPar
Mondi 3
\\
\sphinxhline
\sphinxAtStartPar
\sphinxstyleemphasis{Oct 17}
&
\sphinxAtStartPar
\sphinxstyleemphasis{Tue}
&
\sphinxAtStartPar
\sphinxcode{\sphinxupquote{10.15am\sphinxhyphen{}11.30am}}
&
\sphinxAtStartPar
Lecture
&
\sphinxAtStartPar
Mondi 3
\\
\sphinxhline
\sphinxAtStartPar
\sphinxstyleemphasis{Oct 17}
&
\sphinxAtStartPar
\sphinxstyleemphasis{Tue}
&
\sphinxAtStartPar
\sphinxcode{\sphinxupquote{11.45am\sphinxhyphen{}12.35pm}}
&
\sphinxAtStartPar
Recitation
&
\sphinxAtStartPar
Mondi 3
\\
\sphinxhline
\sphinxAtStartPar
\sphinxstylestrong{Oct 18}
&
\sphinxAtStartPar
\sphinxstylestrong{Wed}
&
\sphinxAtStartPar
\sphinxcode{\sphinxupquote{11.30am\sphinxhyphen{}12.45pm}}
&
\sphinxAtStartPar
Lecture
&
\sphinxAtStartPar
\sphinxstylestrong{Mondi 1}
\\
\sphinxhline
\sphinxAtStartPar
\sphinxstyleemphasis{Oct 24}
&
\sphinxAtStartPar
\sphinxstyleemphasis{Tue}
&
\sphinxAtStartPar
\sphinxcode{\sphinxupquote{10.15am\sphinxhyphen{}11.30am}}
&
\sphinxAtStartPar
Lecture
&
\sphinxAtStartPar
Mondi 3
\\
\sphinxhline
\sphinxAtStartPar
\sphinxstyleemphasis{Oct 24}
&
\sphinxAtStartPar
\sphinxstyleemphasis{Tue}
&
\sphinxAtStartPar
\sphinxcode{\sphinxupquote{11.45am\sphinxhyphen{}12.35pm}}
&
\sphinxAtStartPar
Recitation
&
\sphinxAtStartPar
Mondi 3
\\
\sphinxhline
\sphinxAtStartPar
\sphinxstylestrong{Oct 25}
&
\sphinxAtStartPar
\sphinxstylestrong{Wed}
&
\sphinxAtStartPar
\sphinxcode{\sphinxupquote{11.30am\sphinxhyphen{}12.45pm}}
&
\sphinxAtStartPar
Lecture
&
\sphinxAtStartPar
\sphinxstylestrong{Mondi 1}
\\
\sphinxhline
\sphinxAtStartPar
\sphinxstyleemphasis{Nov 2}
&
\sphinxAtStartPar
\sphinxstyleemphasis{Thu}
&
\sphinxAtStartPar
\sphinxcode{\sphinxupquote{11.20am–12.35pm}}
&
\sphinxAtStartPar
Lecture
&
\sphinxAtStartPar
Mondi 3
\\
\sphinxhline
\sphinxAtStartPar
\sphinxstyleemphasis{Nov 7}
&
\sphinxAtStartPar
\sphinxstyleemphasis{Tue}
&
\sphinxAtStartPar
\sphinxcode{\sphinxupquote{10.15am\sphinxhyphen{}11.30am}}
&
\sphinxAtStartPar
Lecture
&
\sphinxAtStartPar
Mondi 3
\\
\sphinxhline
\sphinxAtStartPar
\sphinxstyleemphasis{Nov 7}
&
\sphinxAtStartPar
\sphinxstyleemphasis{Tue}
&
\sphinxAtStartPar
\sphinxcode{\sphinxupquote{11.45am\sphinxhyphen{}12.35pm}}
&
\sphinxAtStartPar
Recitation
&
\sphinxAtStartPar
Mondi 3
\\
\sphinxhline
\sphinxAtStartPar
\sphinxstyleemphasis{Nov 9}
&
\sphinxAtStartPar
\sphinxstyleemphasis{Thu}
&
\sphinxAtStartPar
\sphinxcode{\sphinxupquote{11.20am–12.35pm}}
&
\sphinxAtStartPar
Lecture
&
\sphinxAtStartPar
Mondi 3
\\
\sphinxhline
\sphinxAtStartPar
\sphinxstyleemphasis{Nov 14}
&
\sphinxAtStartPar
\sphinxstyleemphasis{Tue}
&
\sphinxAtStartPar
\sphinxcode{\sphinxupquote{10.15am\sphinxhyphen{}11.30am}}
&
\sphinxAtStartPar
Lecture
&
\sphinxAtStartPar
Mondi 3
\\
\sphinxhline
\sphinxAtStartPar
\sphinxstyleemphasis{Nov 14}
&
\sphinxAtStartPar
\sphinxstyleemphasis{Tue}
&
\sphinxAtStartPar
\sphinxcode{\sphinxupquote{11.45am\sphinxhyphen{}12.35pm}}
&
\sphinxAtStartPar
Recitation
&
\sphinxAtStartPar
Mondi 3
\\
\sphinxhline
\sphinxAtStartPar
\sphinxstyleemphasis{Nov 16}
&
\sphinxAtStartPar
\sphinxstyleemphasis{Thu}
&
\sphinxAtStartPar
\sphinxcode{\sphinxupquote{11.20am–12.35pm}}
&
\sphinxAtStartPar
Lecture
&
\sphinxAtStartPar
Mondi 3
\\
\sphinxhline
\sphinxAtStartPar
\sphinxstyleemphasis{Nov 21}
&
\sphinxAtStartPar
\sphinxstyleemphasis{Tue}
&
\sphinxAtStartPar
\sphinxcode{\sphinxupquote{10.15am\sphinxhyphen{}11.30am}}
&
\sphinxAtStartPar
Lecture
&
\sphinxAtStartPar
Mondi 3
\\
\sphinxhline
\sphinxAtStartPar
\sphinxstyleemphasis{Nov 21}
&
\sphinxAtStartPar
\sphinxstyleemphasis{Tue}
&
\sphinxAtStartPar
\sphinxcode{\sphinxupquote{11.45am\sphinxhyphen{}12.35pm}}
&
\sphinxAtStartPar
Recitation
&
\sphinxAtStartPar
Mondi 3
\\
\sphinxhline
\sphinxAtStartPar
\sphinxstyleemphasis{Nov 23}
&
\sphinxAtStartPar
\sphinxstyleemphasis{Thu}
&
\sphinxAtStartPar
\sphinxcode{\sphinxupquote{11.20am–12.35pm}}
&
\sphinxAtStartPar
Lecture
&
\sphinxAtStartPar
Mondi 3
\\
\sphinxbottomrule
\end{tabulary}
\sphinxtableafterendhook\par
\sphinxattableend\end{savenotes}


\section{Content}
\label{\detokenize{teaching/random_matrices_2017:content}}

\subsection{October 12}
\label{\detokenize{teaching/random_matrices_2017:october-12}}\begin{enumerate}
\sphinxsetlistlabels{\arabic}{enumi}{enumii}{}{.}%
\item {} 
\sphinxAtStartPar
Basic facts from probability theory. Law of large numbers (LLN) and the central limit theorem (CLT), viewed as universality statements. In the LLN the limit is deterministic, while in CLT the limit is a random variable, namely the Gaussian (normal) one. No matter which distribution the initial random variables had, their appropriately normalized sums always converge to the same distribution — in other words the limiting Gaussian distribution is universal.

\item {} 
\sphinxAtStartPar
Wigner random matrices. Real symmetric and complex hermitian. GUE and GOE.  Wishart matrices and their relation to Wigner\sphinxhyphen{}type matrices. Scaling so that eigenvalues remain bounded. Statement on the concentration of the largest eigenvalue. Introducing the semicircle law as a law of large numbers for the empirical density of the eigenvalues.

\item {} 
\sphinxAtStartPar
Linear statistics of eigenvalues (with a smooth function as observable) leads to CLT but with an unusual scaling — indicating very strong correlation among eigenvalues.

\item {} 
\sphinxAtStartPar
Statement of the gap universality, Wigner surmise. The limit behavior of the gap is a new universal distribution; in this sense this is the analogue of the CLT.

\end{enumerate}

\sphinxAtStartPar
\sphinxstylestrong{Reading.} \DUrole{xref,download,myst}{PCMI lecture notes} up to middle of Section 1.2.3.


\subsection{October 17}
\label{\detokenize{teaching/random_matrices_2017:october-17}}\begin{enumerate}
\sphinxsetlistlabels{\arabic}{enumi}{enumii}{}{.}%
\item {} 
\sphinxAtStartPar
Main questions in random matrix theory:
\begin{itemize}
\item {} 
\sphinxAtStartPar
Density on global scale (like LLN)

\item {} 
\sphinxAtStartPar
Extreme eigenvalues (especially relevant for sample covariance matrices)

\item {} 
\sphinxAtStartPar
Fluctuation on the scale of eigenvalue spacing (like CLT)

\item {} 
\sphinxAtStartPar
Mesoscopic density — follows the global behaviour, but it is a non\sphinxhyphen{}trivial fact.

\item {} 
\sphinxAtStartPar
Eigenfunction (de)localization

\end{itemize}

\item {} 
\sphinxAtStartPar
Definition of \(k\)\sphinxhyphen{}point correlation functions. Relation of the gap distribution to the local correlation functions on scale of the eigenvalue spacing (inclusion\sphinxhyphen{}exclusion formula)

\item {} 
\sphinxAtStartPar
Rescaled (local) correlation functions. Determinant structure. Sine kernel for complex Hermitian Wigner matrices. Statement of the main universality result in the bulk spectrum (for energies away from the edges of the semicircle law).

\end{enumerate}

\sphinxAtStartPar
\sphinxstylestrong{Reading.} \DUrole{xref,download,myst}{PCMI lecture notes} up to the end of Section 1.2.3.


\subsection{October 17 (Recitation)}
\label{\detokenize{teaching/random_matrices_2017:october-17-recitation}}\begin{enumerate}
\sphinxsetlistlabels{\arabic}{enumi}{enumii}{}{.}%
\item {} 
\sphinxAtStartPar
Definition of Stieltjes transform
\begin{equation}\label{equation:teaching/random_matrices_2017:eq-Stieltjes}
\begin{split}
    m_\mu(z):=\int_{\mathbb R} \frac{d\mu(\lambda)}{\lambda-z},\quad z\in\mathbb{H}:=\{z\in\mathbb C,\, \Im z>0\}
    \end{split}
\end{equation}
\sphinxAtStartPar
of probability measure \(\mu\) and statement of elementary properties (analyticity, trivial bounds on derivatives). Interpretation of the Stieltjes transform of the empirical spectral density as the normalized trace of the resolvent.

\item {} 
\sphinxAtStartPar
Interpretation of the imaginary part of \eqref{equation:teaching/random_matrices_2017:eq-Stieltjes} as the convolution with the Poisson kernel,
\begin{equation*}
\begin{split}
    \Im m_\mu(x+i\eta)= \pi (P_\eta\ast \mu)(x),\quad P_\eta(x):=\frac{1}{\pi}\frac{\eta}{x^2+\eta^2}.
    \end{split}
\end{equation*}
\sphinxAtStartPar
The Stieltjes transform \(m_\mu(x+i\eta)\) thus contains information about \(\mu\) at a scale of \(\eta\) around \(x\).

\item {} 
\sphinxAtStartPar
\sphinxstyleemphasis{Stieltjes continuity theorem for sequences of random measures:} A sequence of random probability measures \(\mu_1,\mu_2,\dots\) converges vaguely, a) in expectation b) in probability c) almost surely to a deterministic probability measure \(\mu\) if and only if for all \(z\in\mathbb H\), the sequence of numbers \(m_{\mu_N}(z)\) converges a) in expectation b) in probability c) almost surely to \(m_{\mu}(z)\).

\item {} 
\sphinxAtStartPar
Derivation of the \sphinxstyleemphasis{Helffer\sphinxhyphen{}Sjöstrand formula}
\begin{equation*}
\begin{split}
    f(\lambda)=\frac{1}{2\pi i}\int_{\mathbb C} \frac{\partial_{\overline z} f_{\mathbb C} (z)}{\lambda-z}d \overline z \wedge d z,\quad f_{\mathbb C}(x+i\eta):= \chi(\eta)\big[f(x)+i\eta f'(x)\big]
    \end{split}
\end{equation*}
\end{enumerate}

\sphinxAtStartPar
for compactly supported \(C^2\)\sphinxhyphen{}functions \(f\colon\mathbb R\to\mathbb R\) and some smooth cut\sphinxhyphen{}off function \(\chi\).


\subsection{October 18}
\label{\detokenize{teaching/random_matrices_2017:october-18}}
\sphinxAtStartPar
\sphinxstylestrong{Main motivations for random matrices:}
\begin{enumerate}
\sphinxsetlistlabels{\arabic}{enumi}{enumii}{}{.}%
\item {} 
\sphinxAtStartPar
Wigner’s original motivation: to model energy levels of heavy nuclei. The distribution of the gaps very well matched that of the Wigner random matrices. The density of states depends on the actual nucleus (and it is not the semicircle), but the local statistics (e.g. gap statistics) are universal.

\item {} 
\sphinxAtStartPar
Random Schrodinger operators, Anderson transition

\item {} 
\sphinxAtStartPar
Gap statistics of the zeros of the Riemann zeta function.

\end{enumerate}

\sphinxAtStartPar
\sphinxstylestrong{Quantum Mechanics in nutshell:}
\begin{itemize}
\item {} 
\sphinxAtStartPar
Configuration space: \(S\) (with a measure)

\item {} 
\sphinxAtStartPar
State space: \(\ell^2(S)\) (square integrable functions on \(S\))

\item {} 
\sphinxAtStartPar
Observables: self\sphinxhyphen{}adjoint (symmetric) operators on \(\ell^2(S)\)

\item {} 
\sphinxAtStartPar
A distinguished observable: the Hamilton (or energy) operator

\item {} 
\sphinxAtStartPar
Time evolution — Schrödinger equation.

\end{itemize}

\sphinxAtStartPar
Random Schrödinger operator describes a single electron in an ionic (metallic) lattice. \(S = \mathbb Z^d\) or a subset of that. \(H\) is the sum of the discrete (lattice) Laplace operator and a random potential.

\sphinxAtStartPar
\sphinxstylestrong{Anderson phase transition:} depending on the strength of the disorder, the system is either in delocalized (conductor) or localized (insulator) phase. Localized phase is characterized by
\begin{itemize}
\item {} 
\sphinxAtStartPar
Localized eigenfunctions

\item {} 
\sphinxAtStartPar
Localized time evolution (no transport)

\item {} 
\sphinxAtStartPar
Pure point spectrum (for the infinite volume operator)

\item {} 
\sphinxAtStartPar
Poisson local spectral statistics, no level repulsion (for the finite volume model)

\end{itemize}

\sphinxAtStartPar
In the delocalized phase, we have delocalized eigenfunctions (“almost” \(\ell^2\)\sphinxhyphen{}normalizable solutions to the eigenvalue equation), quantum transport, absolutely continuous spectrum and random matrix eigenvalue statistics, in particular level repulsion.

\sphinxAtStartPar
\sphinxstylestrong{Reading.} \DUrole{xref,download,myst}{PCMI lecture notes} Sections 5.1 — 5.3


\subsection{October 24}
\label{\detokenize{teaching/random_matrices_2017:october-24}}
\sphinxAtStartPar
Phase diagram for the Anderson model (= random Schrödinger operator on the \(\mathbb Z^d\) lattice) in \(d\ge 3\) dimensions. Localized regime can be proven, delocalized regime is conjectured to exist but no mathematical result.

\sphinxAtStartPar
In \(d=1\) dimension the Anderson model is always localized (transfer matrix method helps). In \(d=2\) nothing is known, even there is no clear agreement in the physics whether it behaves more like \(d=1\) (localization) or more like \(d=3\) (delocalization); majority believes in localization.

\sphinxAtStartPar
Delocalized regime, at least for small disorder, sounds easier to prove because it looks like a perturbative problem (zero disorder corresponds to the pure Laplacian which is perfectly understood). Resolvent perturbation formulas were discussed; major problem: lack of convergence.

\sphinxAtStartPar
We gave some explanation why the localization regime is easier to handle mathematically: off\sphinxhyphen{}diagonal resolvent matrix elements decay exponentially. This fact provides an effective decoupling and makes localized resolvents almost independent.

\sphinxAtStartPar
Random band matrices: naturally interpolate between \(d=1\) dimensional random Schrödinger operators (bandwidth \(W=O(1)\)) and mean field Wigner matrices (bandwidth \(W = N\), where \(N\) is the linear size of the system). Phase transition is expected at \(W = \sqrt{N}\); this is a major open question. There are similar conjectures in higher dimensional band matrices, but we did not discuss them.

\sphinxAtStartPar
Finally, we discussed a mysterious connection between the Dyson sine kernel statistics and the location of the zeros of of the zeta function on the critical line. There is only one mathematical result in this direction, Montgomery proved that the two point function of the (appropriately rescaled) zeros follows the sine kernel behavior, but only for test functions with Fourier support in \([-1,1]\). No progress has been made in the last 40 years to relax this condition.

\sphinxAtStartPar
\sphinxstylestrong{Reading.} \DUrole{xref,download,myst}{PCMI lecture notes} Section 5.3 and the entertaining article \sphinxhref{http://www.cims.nyu.edu/~bourgade/papers/TeaTime.pdf}{“Tea Time in Princeton”} by Paul Bourgade about Montgomery’s theorem.


\subsection{October 24 (Recitation)}
\label{\detokenize{teaching/random_matrices_2017:october-24-recitation}}\begin{enumerate}
\sphinxsetlistlabels{\arabic}{enumi}{enumii}{}{.}%
\item {} 
\sphinxAtStartPar
Analytic definition of (multivariate) cumulants \(\kappa_\alpha\) of a random vector \(X=(X_1,\dots,X_n)\) as the coefficients of the log\sphinxhyphen{}characteristic function
\begin{equation*}
\begin{split}
    \log \mathbf E e^{i t\cdot X} = \sum_\alpha \kappa_\alpha \frac{(it)^\alpha}{\alpha!}.
    \end{split}
\end{equation*}
\item {} 
\sphinxAtStartPar
Proof of the cumulant expansion formula
\begin{equation*}
\begin{split}
    \mathbf E X_i f(X)=\sum_{\alpha} \frac{\kappa_{\alpha, i }}{\alpha!}\mathbf E f^{(\alpha)}(X)
    \end{split}
\end{equation*}
\sphinxAtStartPar
via Fourier transform.

\item {} 
\sphinxAtStartPar
Expression of moments in terms of cumulants as the sum of all partitions
\begin{equation*}
\begin{split}
    \mathbf{E} X_1\dots X_n=\sum_{\mathcal{P}\vdash [n]} \kappa^{\mathcal{P}}=\sum_{\mathcal{P}\vdash [n]}\prod_{P_i\in\mathcal{P}} \kappa( X_j \mid j\in P_i )
    \end{split}
\end{equation*}
\item {} 
\sphinxAtStartPar
Derivation of the inverse relationship
\begin{equation}\label{equation:teaching/random_matrices_2017:eq-combcum}
\begin{split}
    \kappa(X_1,\dots,X_n)=\sum_{\mathcal P\vdash [n]}(-1)^{\lvert\mathcal P\rvert-1}(\lvert\mathcal P\rvert-1)! \prod_{P_i\in\mathcal P} \mathbf E \prod_{j\in P_i} X_j
    \end{split}
\end{equation}
\sphinxAtStartPar
through \sphinxhref{https://en.wikipedia.org/wiki/Incidence\_algebra}{Möbius inversion} on abstract incidence algebras. Note that \eqref{equation:teaching/random_matrices_2017:eq-combcum} can also serve as a purely combinatorial definition of cumulants.

\item {} 
\sphinxAtStartPar
Proof that cumulants of random variables which split into two independent subgroups vanish.

\end{enumerate}


\subsection{October 25}
\label{\detokenize{teaching/random_matrices_2017:october-25}}
\sphinxAtStartPar
There are two natural ways to put a measure on the space of (hermitian) matrices, hence defining two major classes of random matrix ensembles:
\begin{enumerate}
\sphinxsetlistlabels{\arabic}{enumi}{enumii}{}{.}%
\item {} 
\sphinxAtStartPar
Choose matrix elements independently (modulo the hermitian symmetry) from some distribution on the complex or real numbers. This results in Wigner matrices (and possible generalizations, when identicality of the distribution is dropped).

\item {} 
\sphinxAtStartPar
Equip the space of hermitian matrices with the usual Lebesgue measure and multiply it by a Radon\sphinxhyphen{}Nikodym factor that makes the measure finite. We choose the factor invariant under unitary conjugation in the form \(\exp(-\text{Tr}\, V(H))\) for some real valued function \(V\). These are called invariant ensembles.

\end{enumerate}

\sphinxAtStartPar
Only Gaussian matrices belong to both families.

\sphinxAtStartPar
For invariant ensembles, the joint probability density function of the eigenvalues can be computed explicitly and it consists of the Vandermonde determinant (to the first or second power, \(\beta=1,2\), depending on the symmetry class). We sketched of the proof by change of variables.

\sphinxAtStartPar
Invariant ensembles can also be represented as Gibbs measure of N points on the real line with a one\sphinxhyphen{}body potential \(V\) and a logarithmic two\sphinxhyphen{}body interaction. This interpretation allows for choosing any \(\beta>0\), yielding the beta\sphinxhyphen{}ensembles, even though there is no matrix or eigenvalues behind them. There are analogous universality statements for beta\sphinxhyphen{}ensembles, which assert that the local statistics depend only on the parameter beta and are independent of the potential \(V\).

\sphinxAtStartPar
\sphinxstylestrong{Reading.} \DUrole{xref,download,myst}{PCMI lecture notes} Section 1.1.2


\subsection{November 2}
\label{\detokenize{teaching/random_matrices_2017:november-2}}
\sphinxAtStartPar
Precise statement of the Wigner semicircle law (for i.i.d. case) in the form of weak convergence in probability. In general, there are two methods to prove the semicircle law:
\begin{enumerate}
\sphinxsetlistlabels{\arabic}{enumi}{enumii}{}{.}%
\item {} 
\sphinxAtStartPar
Moment method: computes \(\text{Tr}\, H^k\), obtains the distribution of the moments of the eigenvalues. The moments are given by the \sphinxhref{https://en.wikipedia.org/wiki/Catalan\_number}{Catalan numbers} and they uniquely identify the semicircle law (calculus exercise) using \sphinxhref{https://en.wikipedia.org/wiki/Carleman\%27s\_condition}{Carleman theorem} on the uniqueness of the measure if the moments do not grow too fast.

\item {} 
\sphinxAtStartPar
Resolvent method: derives an equation for the limiting Stieltjes transform of the empirical density.

\end{enumerate}

\sphinxAtStartPar
The resolvent method in general is more powerful, it works well inside as well as neat the edge of the spectrum. The moment method is powerful only at the extreme edges.

\begin{sphinxadmonition}{note}{Proof of the Wigner semicircle law by moment method}

\sphinxAtStartPar
Compute
\begin{equation*}
\begin{split}
\frac{1}{N} \mathbb E \text{Tr}\, H^k=\frac{1}{N}\mathbb E\sum_{i_1,\dots,i_k} h_{i_1i_2}h_{i_2i_3}\dots h_{i_{k-1}i_k}h_{i_ki_1}
\end{split}
\end{equation*}
\sphinxAtStartPar
in terms of the number of backtracking paths (only those path give a relevant contribution where every edge is travelled exactly twice and the skeleton of the graph is a tree). We reduced the problem to counting such path — it is an \(N\) independent problem.
\end{sphinxadmonition}


\subsection{November 7}
\label{\detokenize{teaching/random_matrices_2017:november-7}}
\sphinxAtStartPar
We completed the proof of the Wigner semicircle law by moment method. Last time we showed that to evaluate \(\mathbb E \text{Tr}\, H^{2k}\) is sufficient to count the number of backtracking path of total length \(2k\). This number has many other combinatorial interpretations. It is the same as the number of rooted, oriented trees on \(k+1\) vertices by a simple one to one correspondance. It is also the same as the number of Dyck paths of length \(2k\), where a Dyck path is a random walk on the nonnegative integers starting and ending at \(0\). Finally, we counted the Dyck paths by deriving the recursion
\begin{equation*}
\begin{split}
C_k = C_{k-1} C_0 + C_{k-2} C_1 + … + C_0 C_{k-1}
\end{split}
\end{equation*}
\sphinxAtStartPar
with \(C_0=1\) for their number \(C_k\). This recursion can be solved by considering the generating
function
\begin{equation*}
\begin{split}
f(x) = \sum_{k=0}^\infty C_k x^k
\end{split}
\end{equation*}
\sphinxAtStartPar
and observe that
\begin{equation*}
\begin{split}
xf^2(x) = f(x) - 1.
\end{split}
\end{equation*}
\sphinxAtStartPar
Thus \(f(x)\) can be explicitly computed by the solution formula for the quadratic equation and Taylor expanding around \(x=0\). After some calculation with the fractional binomial coefficients, we obtain that \(C_k = 1/(k+1) {2k \choose k}\), i.e. the Catalan numbers.

\sphinxAtStartPar
Since the Catalan numbers are the moments of the semicircle law (calculus exercise), and these moments do not grow too fast, they identify the measure.

\sphinxAtStartPar
This proved that the expectation of the empirical eigenvalue density converges to the semicircle in the sense of moments. Using compact support of the measures (for the empirical density we know it from the homework problem since the norm of \(H\) is bounded), by Weierstrass theorem we can extend the
convergence for any bounded continuous functions.

\sphinxAtStartPar
Finally, the expectation can be removed, by computing the variance of
\(N^{-1} \text{Tr}\, H^k\), again by the graphical representation (now we have two cycles and studied which edge\sphinxhyphen{}coincidences give rise to nonzero contribution). We showed that the variance vanishes in the large \(N\) limit and then a Chebyshev inequality converts it into a high probability bound.

\sphinxAtStartPar
\sphinxstylestrong{Reading.} \DUrole{xref,download,myst}{PCMI lecture notes} Section 2.3


\subsection{November 7 (Recitation)}
\label{\detokenize{teaching/random_matrices_2017:november-7-recitation}}
\sphinxAtStartPar
We found yet another combinatorial description of Catalan numbers. \(C_k\) is the number of non\sphinxhyphen{}crossing pair partitions of the set \(\{1,\dots,2k\}\). Indeed, denote the number in question by \(N_k\). Then there exists some \(j\) such that \(1\) is paired with \(2j\) since due to the absence of crossings there has to be an even number of other integers between \(1\) and its partner. The number of non\sphinxhyphen{}crossing pairings of the integers \(\{2,\dots,2j-1\}\) and \(\{2j+1,\dots,2k\}\) are given by \(N_{j-1}\) and \(N_{k-j}\) respectively and it follows that
\begin{equation*}
\begin{split}
N_{k}=\sum_{j=1}^k N_{j-1}N_{k-j}, \qquad N_1=1
\end{split}
\end{equation*}
and thus \(N_k=C_k\) since they satisfy the same recursion.

\sphinxAtStartPar
We defined a commonly used notion of stochastic domination \(X\prec Y\) and stated the following \sphinxstyleemphasis{large deviation estimates} for families of random variables \(X_i,Y_i\) of zero mean \(\mathbf E X_i=\mathbf E Y_i=0\) and unit variance \(\mathbf E \lvert X_i\rvert^2=\mathbf E \lvert Y_i\rvert^2=1\) and deterministic coefficients \(b_i\), \(a_{ij}\),
\begin{equation*}
\begin{split}
\left\lvert\sum_{i} b_i X_i\right\rvert\prec \left(\sum_i\lvert b_i\rvert^2\right)^{1/2}
\end{split}
\end{equation*}\begin{equation}\label{equation:teaching/random_matrices_2017:eq-LDE}
\begin{split}
\left\lvert\sum_{i,j} a_{ij} X_i Y_j\right\rvert\prec \left(\sum_{i,j}\lvert a_{ij}\rvert^2\right)^{1/2}
\end{split}
\end{equation}\begin{equation*}
\begin{split}
\left\lvert\sum_{i\not=j} a_{ij} X_i X_j\right\rvert\prec \left(\sum_{i\not=j}\lvert a_{ij}\rvert^2\right)^{1/2}
\end{split}
\end{equation*}
\sphinxAtStartPar
We proved \eqref{equation:teaching/random_matrices_2017:eq-LDE} only for uniformly subgaussian families of random variables but not that uniformly finite moments of all orders are also sufficient for them to hold.


\subsection{November 9}
\label{\detokenize{teaching/random_matrices_2017:november-9}}\begin{itemize}
\item {} 
\sphinxAtStartPar
Precise statement of the local semicircle laws (entrywise, isotropic, averaged) for Wigner type matrices with moment condition of arbitrary order.

\item {} 
\sphinxAtStartPar
Definition of stochastic dominations, some properties.

\item {} 
\sphinxAtStartPar
We started the proof of the weak law for Wigner matrices.

\item {} 
\sphinxAtStartPar
Schur complement formula. Almost selfconsistent equation for \(m_N = N^{-1} \text{Tr}\, G\) assuming that the fluctuation of the quadratic term is small (will be proven later).

\item {} 
\sphinxAtStartPar
The other two errors were shown to be small. The smallness of the single diagonal element \(h_{ii}\) directly follows from the moment condition. The difference of the Stieltjes transform of the resolvent and its minor was estimated via interlacing and integration by parts.

\end{itemize}

\sphinxAtStartPar
\sphinxstylestrong{Reading.} \DUrole{xref,download,myst}{PCMI lecture notes} Section 3.1.1.


\subsection{November 14}
\label{\detokenize{teaching/random_matrices_2017:november-14}}\begin{itemize}
\item {} 
\sphinxAtStartPar
Proof of the weak local law in the bulk.

\item {} 
\sphinxAtStartPar
Stability of the equation for \(m_{sc}\), the Stieltjes transform of the semicircle law.

\item {} 
\sphinxAtStartPar
Proof for the large eta regime.

\item {} 
\sphinxAtStartPar
Breaking the circularity of the argument in two steps: In the first step one proves a weaker bound that allows one to approximate \(m\) via \(m_{sc}\), then run the argument again but with improved inputs. The bootstrap argument will have the same philosophy next time.

\item {} 
\sphinxAtStartPar
Discussion of the uniformity in the spectral parameter. Grid argument to improve the bound for supremum over all \(z\). This argument works because (i) the probabilistic bound for any fixed \(z\) is very strong (arbitrary \(1/N\)\sphinxhyphen{}power) and (ii) the function we consider \((m-m_{sc})(z)\) has some weak but deterministic Lipschitz continuity.

\end{itemize}


\subsection{November 14 (Recitation)}
\label{\detokenize{teaching/random_matrices_2017:november-14-recitation}}
\sphinxAtStartPar
We presented a cumulant approach to proving local laws for correlated random matrices. Specifically, we gave a heuristic argument that the resolvent \(G\) should be well apprixmated by the unique solution \(M=M(z)\) to the matrix Dyson equation (MDE)
\begin{equation*}
\begin{split}
0=1+zM+\mathcal S[M]M, \quad \Im M>0,\qquad \mathcal S[R]:= \sum_{\alpha,\beta}\text{Cov}(h_\alpha,h_\beta) \Delta^\alpha R\Delta^\beta.
\end{split}
\end{equation*}
We furthermore proved that the error matrix
\begin{equation*}
\begin{split}
D=1+zG+\mathcal S[G]G=HG+\mathcal S[G]G
\end{split}
\end{equation*}
satisfies
\begin{equation*}
\begin{split}
\mathbf E\lvert\langle x,Dy \rangle\rvert^2 \lesssim \left(\frac{\lVert x\rVert \lVert y\rVert}{\sqrt{N\eta}}\right)^2,\qquad \mathbf E\lvert\langle BD \rangle\rvert^2 \lesssim \left(\frac{\lVert B\rVert}{N\eta}\right)^2
\end{split}
\end{equation*}
in the case of Gaussian entries \(h_\alpha\).


\subsection{November 16}
\label{\detokenize{teaching/random_matrices_2017:november-16}}\begin{itemize}
\item {} 
\sphinxAtStartPar
We completed the rigorous proof of the weak local semicircle law
by the bootstrap argument.

\item {} 
\sphinxAtStartPar
Then we mentioned two improvements: (i) Strong local law (error bound improved from \((N \eta)^{-1/2}\) to \((N\eta)^{-1}\) and (ii) entrywise local law.

\item {} 
\sphinxAtStartPar
Proof of the entrywise local law via the self\sphinxhyphen{}consistent vector equation. Stability operator mentioned in the more general setup of Wigner type matrices (when the variance matrix \(S\) is stochastic). Diagonal and offdiagonal elements are estimated separately via a joint control parameter \(\Lambda\). Main ideas are sketched, the rigorous bootstrap argument was omitted.

\end{itemize}

\sphinxAtStartPar
\sphinxstylestrong{Reading.} \DUrole{xref,download,myst}{PCMI lecture notes} Sections 4.1–4.3


\subsection{November 21}
\label{\detokenize{teaching/random_matrices_2017:november-21}}\begin{itemize}
\item {} 
\sphinxAtStartPar
Fluctuation averaging phenomenon. Proof of the strong local law in the bulk. Some remarks on the modifications at the edge. Corollaries of the strong local law: optimal estimates on the eigenvalue counting function and rigidity (location of the individual eigenvalues).

\item {} 
\sphinxAtStartPar
Bulk universality for Hermitian Wigner matrices. Basic idea: interpolation. Ornstein Uhlenbeck process for matrices (preserves expectation and variance). Crash course on Brownian motion, stochastic integration and Ito’s formula. Dyson Brownian motion (DBM) for the eigenvalues. Local equilibration phenomenon due to the strong level repulsion in the DBM.

\end{itemize}


\subsection{November 23}
\label{\detokenize{teaching/random_matrices_2017:november-23}}
\sphinxAtStartPar
We summarized the three step strategy to prove local spectral universality of Wigner matrices. We discussed the second step: fast convergence to equilibrium of the Dyson Brownian Motion. Relation between SDE and PDE: introduction of the generator. Laplacian is the generator of the standard Brownian motion.

\sphinxAtStartPar
Basics of large dimensional analysis: Gibbs measure, entropy, Dirichlet form, generator. The total mass of a probability measure is preserved under the dynamics. Relation between various concepts of closeness to equilibrium. Entropy inequality (total variation norm is bounded by the entropy). Logarithmic Sobolev inequality. Spectral gap inequality. Bakry\sphinxhyphen{}Emery theory: (i) the Gibbs measure with a convex Hamiltonian satisfies LSI, (ii) entropy and
Dirichlet form decays exponentially fast.


\section{Homework}
\label{\detokenize{teaching/random_matrices_2017:homework}}
\sphinxAtStartPar
The problem sheets can either be handed in during the lecture or put in the letter box of Dominik Schröder in LBW, 3rd floor.


\begin{savenotes}\sphinxattablestart
\sphinxthistablewithglobalstyle
\centering
\begin{tabulary}{\linewidth}[t]{TTTT}
\sphinxtoprule

\sphinxAtStartPar

&
\sphinxAtStartPar

&\sphinxstyletheadfamily 
\sphinxAtStartPar
published
&\sphinxstyletheadfamily 
\sphinxAtStartPar
due
\\
\sphinxmidrule
\sphinxtableatstartofbodyhook
\sphinxAtStartPar
\DUrole{xref,download,myst}{Problem sheet I}
&
\sphinxAtStartPar
\DUrole{xref,download,myst}{Solutions}
&
\sphinxAtStartPar
Oct 18
&
\sphinxAtStartPar
Oct 25
\\
\sphinxhline
\sphinxAtStartPar
\DUrole{xref,download,myst}{Problem sheet II}
&
\sphinxAtStartPar
\DUrole{xref,download,myst}{Solutions}
&
\sphinxAtStartPar
Oct 25
&
\sphinxAtStartPar
Nov 7
\\
\sphinxhline
\sphinxAtStartPar
\DUrole{xref,download,myst}{Problem sheet III}
&
\sphinxAtStartPar
\DUrole{xref,download,myst}{Solutions}
&
\sphinxAtStartPar
Nov 9
&
\sphinxAtStartPar
Nov 21
\\
\sphinxbottomrule
\end{tabulary}
\sphinxtableafterendhook\par
\sphinxattableend\end{savenotes}





\renewcommand{\indexname}{Index}

\end{document}